\documentclass{SFEE}
\usepackage[rmargin=1.5cm]{geometry}
\usepackage[english]{babel}
\usepackage{microtype}
\usepackage{xcolor}
\usepackage{lipsum} % for dummy text only
\usepackage[colorlinks,linkcolor=blue!50!black]{hyperref}
\usepackage{graphicx}
\usepackage{wrapfig}
\usepackage{graphicx}
\usepackage{hyperref}
\usepackage{orcidlink}



\begin{document}
%\twocolumn

\title{Gravity Theory}
\subtitle{Sleep more and move less is good for prevent fat depletion before   spring.}
\shorttitle{Do nothing improve the fat layer}
\author[1,*]{Albert Einstein }
\author[2]{Isaac Newton}
\author[3]{Juan de Anda-Suárez \orcidlink{0000-0003-3728-0459}}
\affil[1]{Princeton}
\affil[2]{Cambridge}
\affil[2]{Tecnológico Nacional de México/ITS Purísima del Rincón.}
\affil[*]{Albert.e@heaven.com}
\maketitle
\begin{abstract}
    \lipsum[5]
\end{abstract}
\begin{keywords}
    fat, hibernation, activiity
\end{keywords}

\section{Introduction}

\yinipar{J}instein's first paper, "Folgerungen aus den Capillaritätserscheinungen" ("Conclusions drawn from the phenomena of capillarity"), in which he proposed a model of intermolecular attraction that he afterwards disavowed as worthless, was published in the journal Annalen der Physik in 1900.[77][78] His 24-page doctoral dissertation also addressed a topic in molecular physics. Titled "Eine neue Bestimmung der Moleküldimensionen \cite{1908JRE.....4..411E}

\begin{figure}[b!]
\includegraphics[width=\linewidth]{example-image-a}
\caption{The A of hibernAtion}
\end{figure}
\lipsum[10-13]
\section{Material and methods}
\lipsum[14-17]
\begin{equation}
 \bar{Nu}=cRe^{n}\left ( \frac{P}{r} \right )^{1/4}
\end{equation}
\section{Results and discussion}
\lipsum[18-25]
\begin{table*}[htb]
     \caption{Table dummy}
    \footnotesize
    \center
    \begin{tabular}{rccc}
    \hline \hline
    Algorithm & Friedman (pvalue 1.6312 ) & Aligned Friedman  (pvalue 3.45e-05 ) & Quade (p-value 5.14e-09 )  \\
    \hline
     SEED & 1.1  & 19.1 & 1.09   \\
     \hline
     PSO  & 2.43 & 60.2 & 2.53   \\
     \hline
     DE   & 2.47 & 57.2 & 2.38  \\
    \hline

    \hline \hline
    \end{tabular}
  
    \end{table*}
\section{Conclusion}
\lipsum[18-27]

\bibliography{biblio} 
\bibliographystyle{sfee}

\section*{Author contribution}

\begin{wrapfigure}{l}{25mm} 
    \includegraphics[width=1in,height=1.25in,clip,keepaspectratio]{example-image-a}
  \end{wrapfigure}\par
  \textbf{Albert Eistein} Einstein's first paper, "Folgerungen aus den Capillaritätserscheinungen" ("Conclusions drawn from the phenomena of capillarity"), in which he proposed a model of intermolecular attraction that he afterwards disavowed as worthless, was published in the journal Annalen der Physik in 1900.[77][78] His 24-page doctoral dissertation also addressed a topic in molecular physics. Titled "Eine neue Bestimmung der Moleküldimensionen"\par

\begin{wrapfigure}{l}{25mm} 
    \includegraphics[width=1in,height=1.25in,clip,keepaspectratio]{example-image-a}
  \end{wrapfigure}\par
  \textbf{Isaac Newton} In 1679, Newton returned to his work on celestial mechanics by considering gravitation and its effect on the orbits of planets with reference to Kepler's laws of planetary motion. This followed stimulation by a brief exchange of letters in 1679–80 with Hooke, who had been appointed to manage the Royal Society's correspondence, and who opened a correspondence intended to elicit contributions from Newton to Royal Society transactions..\par




\end{document}
